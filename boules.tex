%Test des illustration en TiKZ
\documentclass[11pt,a4paper]{beamer}
\usepackage[utf8]{inputenc}
\setbeamertemplate{navigation symbols}{}
\usepackage{amsmath}
\usepackage{amsfonts}
\usepackage{amssymb}
\usepackage[T1]{fontenc}
\usepackage{lmodern}
\usepackage[french]{babel}
\usepackage{color}
\usepackage{array}
\usepackage{braket}
\usepackage{ulem}
\title{Le principe d'équivalence \\ \textit{``Une histoire de boules''}}
\author{Ludovic Ducobu}
%\subtitle{}
\date[mars 2019]{28 et 29 mars 2019}
\usetheme[navigation,no-subsection]{UMONS}

%%%Permet de mettre la police en serif en mode math uniquement (plus joli)
%%%Attention, si la police en beamer est normalement sans serif c'est parce que ce serait plus facile à lire dans le cas où la résolution de l'écran est pauvre. Ne pas en abuser donc...
\usefonttheme[onlymath]{serif}

\usepackage{multirow}

\usepackage{tikz}
\usetikzlibrary{decorations.pathmorphing,patterns,backgrounds}

\definecolor{vert}{rgb}{0,0.6,0}
\definecolor{rouge}{rgb}{0.7,0,0}



\newcommand{\caneva}[3]{
\draw[color=#3] (-#1/2,-#2/2) -- ++ (0,#2) -- ++ (#1,0) -- ++ (0,-#2) -- cycle;

}

\newcommand{\boule}[2][1]{
  \draw[style={fill=#2}] (0,0) circle(#1);
}



\begin{document}


\begin{frame}
\begin{center}

\begin{tikzpicture}[background rectangle/.style={fill=cyan}, show background rectangle]
\caneva{5}{8}{cyan}

\only<1>{
\begin{scope}[xshift=-1cm,yshift=3.5cm]
\boule[0.2]{brown}
\draw (0,0) node {\tiny $m_1$};
\end{scope}

\begin{scope}[xshift=1cm,yshift=3.5cm]
\boule[0.4]{brown}
\draw (0,0) node {\scriptsize $m_2$};
\end{scope}
}

\only<2>{
\begin{scope}[xshift=-1cm,yshift=3.cm]
\boule[0.2]{brown}
\draw (0,0) node {\tiny $m_1$};
\end{scope}

\begin{scope}[xshift=1cm,yshift=2.5cm]
\boule[0.4]{brown}
\draw (0,0) node {\scriptsize $m_2$};
\end{scope}
}

\only<3>{
\begin{scope}[xshift=-1cm,yshift=2cm]
\boule[0.2]{brown}
\draw (0,0) node {\tiny $m_1$};
\end{scope}

\begin{scope}[xshift=1cm,yshift=0.5cm]
\boule[0.4]{brown}
\draw (0,0) node {\scriptsize $m_2$};
\end{scope}
}

\only<4>{
\begin{scope}[xshift=-1cm,yshift=1cm]
\boule[0.2]{brown}
\draw (0,0) node {\tiny $m_1$};
\end{scope}

\begin{scope}[xshift=1cm,yshift=-1.5cm]
\boule[0.4]{brown}
\draw (0,0) node {\scriptsize $m_2$};
\end{scope}
}

\end{tikzpicture}

\end{center}
\end{frame}

\begin{frame}
\begin{center}

\begin{tikzpicture}[background rectangle/.style={fill=cyan}, show background rectangle]

\only<1>{
\caneva{5}{6.8}{cyan}
\begin{scope}[yshift=3.5cm, color=cyan]
\coordinate (A) at (-1,-0.2);
\coordinate (B) at (1,0.4);
\draw (A) .. controls +(0,-0.7) and  +(0,0.7).. (B);
\end{scope}

\begin{scope}[xshift=-1cm,yshift=3.5cm]
\boule[0.2]{brown}
\draw (0,0) node {\tiny $m_1$};
\end{scope}

\begin{scope}[xshift=1cm,yshift=3.5cm]
\boule[0.4]{brown}
\draw (0,0) node {\scriptsize $m_2$};
\end{scope}
}


\only<2>{
\caneva{5}{6.8}{cyan}
\begin{scope}[yshift=3.5cm]
\coordinate (A) at (-1,-0.2);
\coordinate (B) at (1,0.4);
\draw (A) .. controls +(0,-0.7) and  +(0,0.7).. (B);
\end{scope}

\begin{scope}[xshift=-1cm,yshift=3.5cm]
\boule[0.2]{brown}
\draw (0,0) node {\tiny $m_1$};
\end{scope}

\begin{scope}[xshift=1cm,yshift=3.5cm]
\boule[0.4]{brown}
\draw (0,0) node {\scriptsize $m_2$};
\end{scope}
}

\only<3>{
\caneva{5}{7.4}{cyan}
\begin{scope}[yshift=3.7cm,color=cyan]
\coordinate (A) at (-1,-0.2);
\coordinate (B) at (1,-0.1);
\draw (A) .. controls +(0,-0.7) and  +(0,0.7).. (B);
\end{scope}

\begin{scope}[yshift=3.cm]
\coordinate (A) at (-1,-0.2);
\coordinate (B) at (1,-0.1);
\draw (A) .. controls +(0,-0.7) and  +(0,0.7).. (B);
\end{scope}

\begin{scope}[xshift=-1cm,yshift=3.cm]
\boule[0.2]{brown}
\draw (0,0) node {\tiny $m_1$};
\end{scope}

\begin{scope}[xshift=1cm,yshift=2.5cm]
\boule[0.4]{brown}
\draw (0,0) node {\scriptsize $m_2$};
\end{scope}
}

\only<4>{
\caneva{5}{8}{cyan}
\begin{scope}[yshift=2cm]
\coordinate (A) at (-1,-0.2);
\coordinate (B) at (1,-1.1);
\draw (A) .. controls +(0,-0.7) and  +(0,0.7).. (B);
\end{scope}

\begin{scope}[xshift=-1cm,yshift=2cm]
\boule[0.2]{brown}
\draw (0,0) node {\tiny $m_1$};
\end{scope}

\begin{scope}[xshift=1cm,yshift=0.5cm]
\boule[0.4]{brown}
\draw (0,0) node {\scriptsize $m_2$};
\end{scope}
}

\only<5>{
\caneva{5}{8}{cyan}
\begin{scope}[yshift=1cm]
\coordinate (A) at (-1,-0.2);
\coordinate (B) at (1,-2.1);
\draw (A) .. controls +(0,-0.2) and  +(0,0.2).. (B);
\end{scope}

\begin{scope}[xshift=-1cm,yshift=1cm]
\boule[0.2]{brown}
\draw (0,0) node {\tiny $m_1$};
\end{scope}

\begin{scope}[xshift=1cm,yshift=-1.5cm]
\boule[0.4]{brown}
\draw (0,0) node {\scriptsize $m_2$};
\end{scope}
}

\end{tikzpicture}

\end{center}
\end{frame}


\end{document}